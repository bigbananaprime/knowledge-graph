\cnabstract

近年来,随着服装电子商务和人工智能技术的高速发展,出现了许多新型应用,如服装商品检索和个性化推荐等。然而,随之而来的也是海量的服装数据。面对这些数据,目前市面上所存在的传统意义上的存储技术对于其价值并不能够进行很好的开发和利用,为了充分利用大数据环境下产生的海量服装数据,并提高检索效率,知识图谱被作为一种人工智能前沿技术而受到广泛关注,为我们提供了一种在结构化的方式下储存和管理数据的方法。因此,本文基于目前主流电商网站的服装商品属性,构建了服装名称知识图谱。本文核心内容包括以下两部分:

(1)基于目前主流电商网站的服装商品属性,以天猫、京东、拼多多等电商网站的服装商品数据作为主要的数据来源,通过爬虫技术获取服装商品数据。在Python环境下,借助于Request、Selenium等Python第三方工具库,通过模仿网上的现实用户在电商平台浏览界面中所进行的操作,利用关键字定位和路径选择来找到符合要求的服装商品数据,并访问到服装商品所相对应的商品数据详情页获取记录Url,通过XPath、Css.selector等特殊元素定位方法对网页进行解析,定位所需的服装商品数据在HTML文件中的位置,最后以纯文本形式存储表格数据。

(2)从知识图谱的研究动态的角度出发,结合本体概念和图数据库技术相关理论内容后,基于服装名称的商品属性,提出了一种基于“七步法”的本体模式层构建方法的改进方法,通过使用Python的第三方工具库py2neo对服装名称知识图谱的本体模式层进行构建,随后对所生成的RDF三元组数据使用语义插件,完成RDF三元组数据到图数据结构的映射,最后将服装商品数据存储到Neo4j图数据库中,完成对于女装、男装、童装与鞋子的服装名称知识图谱的构建。
\cnkeyword{selenium,爬虫,Neo4j图数据库,服装名称知识图谱}