%%%%%%%%%%%%%%%英文摘要%%%%%%%%%%%%
\enabstract

In recent years, many new applications have emerged, such as garment product retrieval and personalized recommendations. However, this has also resulted in massive amounts of clothing data. Traditional storage technologies cannot effectively utilize and develop the value of such massive clothing data. To fully utilize the massive clothing data generated in a big data environment and improve retrieval efficiency, knowledge graphs have been widely studied as a cutting-edge technology in artificial intelligence. They provide a method for storing and managing data in a structured way. In response to the above-mentioned problem, this article constructs a clothing name knowledge graph based on garment attributes to solve the aforementioned problem. The core content of this article includes the following two parts:

(1)Based on the attributes of clothing products on mainstream e-commerce websites, such as Tmall, JD.com, and Pinduoduo, crawler technology is used to obtain clothing product data. In the Python environment, with the help of third-party libraries such as Request and Selenium, simulated user operations in the browser are performed by using the keyword search method to search for matching clothing products. The corresponding URL of the product is obtained and recorded by accessing the details page of the product. XPath, CSS.selector and other special element positioning methods are used to parse the webpage and locate the position of the required clothing product data in the HTML file. Finally, table data is stored in plain text format.

(2)From the perspective of research progress in knowledge graph, combined with ontology concepts and graph database technology, an improved method based on the "seven-step method" ontology pattern layer construction method is proposed for commodity attributes based on clothing names. The clothing domain ontology is constructed using the third-party Python library py2neo, and RDF data is generated. After mapping RDF triples to a graph data structure using a semantic plugin, clothing data is stored in the Neo4j graph database, completing the construction of a knowledge graph for clothing names in women's wear, men's wear, children's wear, and shoes.

\enkeyword{selenium,crawler,Neo4j,Knowledge Graph}